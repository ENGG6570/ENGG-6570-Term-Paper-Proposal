\section{Introduction}
Self-Organizing Map (SOM), also know as Kohonen map, is a topological preserving map that can map a higher dimensional space to a lower dimensional space. Along this process, information will be compressed; while, the key parameters in terms of "topological and metric relationships"\cite{Kohonen1998} will be retained. 
\\
There are two steps involved in forming a self-organizing map from a raw input data-set\cite{hebbian2007}, respectively to be 1) \textbf{competition} and 2) \textbf{cooperation}. When a set of data is feed into the system sequentially with random shuffle, for each input data point, \textbf{competition} will take place first and, based on a pre-defined cost function, one of the neurons on the output layer with the minimal cost will be selected as a winner; Following the competition, the  \textbf{cooperation} will then take place. Based on a neighborhood function, the winner together with it's neighbor neurons will proceed the learning; while, the neurons outside of the winner's neighbor zone will gain no learning. The purpose of the cooperation step is to increase the like-hood that if a similar input pattern present again, the same group of neurons will become the winner with a higher possibility. Iterate with this strategy on the input data-set over a suitable period, without supervising (providing error to the system), the output layer will simultaneously form a map that contains the similar topological structure as the input data. 



\begin{figure}[h]
  \centering
  \includegraphics[width=8cm]{Picture/AnminPic.JPG}
  \caption{The illustration of the intermediate step to map K inputs (Robots) to M outputs (Targets)\cite{zhu2006neural}. At a time instance, $i$, K neurons out of these $K*M$ possibilities will be selected as the winner. [\textcolor{red}{I do not really understand this!!!, should we have K winner as we only have K inputs or M winner as each neuron should have same possibility to win???}] }
  \Description{More People are using Maven as build tool than Gradle}
\end{figure}

