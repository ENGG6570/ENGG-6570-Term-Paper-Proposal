%%
%% This is file `sample-manuscript.tex',
%% generated with the docstrip utility.
%%
%% The original source files were:
%%
%% samples.dtx  (with options: `manuscript')
%% 
%% IMPORTANT NOTICE:
%% 
%% For the copyright see the source file.
%% 
%% Any modified versions of this file must be renamed
%% with new filenames distinct from sample-manuscript.tex.
%% 
%% For distribution of the original source see the terms
%% for copying and modification in the file samples.dtx.
%% 
%% This generated file may be distributed as long as the
%% original source files, as listed above, are part of the
%% same distribution. (The sources need not necessarily be
%% in the same archive or directory.)
%%
%% The first command in your LaTeX source must be the \documentclass command.
\documentclass[manuscript,screen,review]{acmart}

%%
%% \BibTeX command to typeset BibTeX logo in the docs
\AtBeginDocument{%
  \providecommand\BibTeX{{%
    \normalfont B\kern-0.5em{\scshape i\kern-0.25em b}\kern-0.8em\TeX}}}

% %% Rights management information.  This information is sent to you
% %% when you complete the rights form.  These commands have SAMPLE
% %% values in them; it is your responsibility as an author to replace
% %% the commands and values with those provided to you when you
% %% complete the rights form.
% \setcopyright{acmcopyright}
% \copyrightyear{2018}
% \acmYear{2018}
% \acmDOI{10.1145/1122445.1122456}

% %% These commands are for a PROCEEDINGS abstract or paper.
% \acmConference[Woodstock '18]{Woodstock '18: ACM Symposium on Neural
%   Gaze Detection}{June 03--05, 2018}{Woodstock, NY}
% \acmBooktitle{Woodstock '18: ACM Symposium on Neural Gaze Detection,
%   June 03--05, 2018, Woodstock, NY}
% \acmPrice{15.00}
% \acmISBN{978-1-4503-XXXX-X/18/06}


%%
%% Submission ID.
%% Use this when submitting an article to a sponsored event. You'll
%% receive a unique submission ID from the organizers
%% of the event, and this ID should be used as the parameter to this command.
%%\acmSubmissionID{123-A56-BU3}

%%
%% The majority of ACM publications use numbered citations and
%% references.  The command \citestyle{authoryear} switches to the
%% "author year" style.
%%
%% If you are preparing content for an event
%% sponsored by ACM SIGGRAPH, you must use the "author year" style of
%% citations and references.
%% Uncommenting
%% the next command will enable that style.
%%\citestyle{acmauthoryear}
\usepackage{hyperref}
%%
%% end of the preamble, start of the body of the document source.
\begin{document}

%%
%% The "title" command has an optional parameter,
%% allowing the author to define a "short title" to be used in page headers.
\title{Dynamic Task Scheduling with Unsupervised Self Organizing Map}

%%
%% The "author" command and its associated commands are used to define
%% the authors and their affiliations.
%% Of note is the shared affiliation of the first two authors, and the
%% "authornote" and "authornotemark" commands
%% used to denote shared contribution to the research.

\author{Zhimo Zhou}
\authornote{Both authors contributed equally to this research.}
\email{zzhou17@uoguelph.ca}
% \orcid{1234-5678-9012}
\affiliation{%
  \institution{Universe of Guelph}
  \streetaddress{500 Stone Rd, E}
  \city{Guelph}
  \state{Ontario}
  \country{Canada}
  \postcode{N2L 3G1}
}  
\author{Uday Singh}
\authornote{Both authors contributed equally to this research.}
\email{usingh03@uoguelph.ca}
% \orcid{1234-5678-9012}
\affiliation{%
  \institution{Universe of Guelph}
  \streetaddress{500 Stone Rd, E}
  \city{Guelph}
  \state{Ontario}
  \country{Canada}
  \postcode{N2L 3G1}
}
\author{Yaowen Mei}
\authornote{Both authors contributed equally to this research.}
\email{ywmei@uoguelph.ca}
% \orcid{1234-5678-9012}
\affiliation{%
  \institution{Universe of Guelph}
  \streetaddress{500 Stone Rd, E}
  \city{Guelph}
  \state{Ontario}
  \country{Canada}
  \postcode{N2L 3G1}
}
%%
%% By default, the full list of authors will be used in the page
%% headers. Often, this list is too long, and will overlap
%% other information printed in the page headers. This command allows
%% the author to define a more concise list
%% of authors' names for this purpose.
% \renewcommand{\shortauthors}{Trovato and Tobin, et al.}

%%
%% The abstract is a short summary of the work to be presented in the
%% article.
\begin{abstract}
Some Abstract I will add later
\end{abstract}

%%
%% The code below is generated by the tool at http://dl.acm.org/ccs.cfm.
%% Please copy and paste the code instead of the example below.
%%
\begin{CCSXML}
<ccs2012>
   <concept>
       <concept_id>10003456.10003457.10003521.10003524</concept_id>
       <concept_desc>Social and professional topics~History of software</concept_desc>
       <concept_significance>500</concept_significance>
       </concept>
   <concept>
       <concept_id>10010520.10010521</concept_id>
       <concept_desc>Computer systems organization~Architectures</concept_desc>
       <concept_significance>500</concept_significance>
       </concept>
 </ccs2012>
\end{CCSXML}

\ccsdesc[500]{Social and professional topics~History of software}
\ccsdesc[500]{Computer systems organization~Architectures}

%%
%% Keywords. The author(s) should pick words that accurately describe
%% the work being presented. Separate the keywords with commas.
\keywords{ Android Studio, Eclipse, Apache Maven, Gradle, Software Architecture}
%%
%% This command processes the author and affiliation and title
%% information and builds the first part of the formatted document.
\maketitle
\section{Introduction}
Self-Organizing Map (SOM), also know as Kohonen map, is a topological preserving map that can map a higher dimensional space to a lower dimensional space. Along this process, information will be compressed; while, the key parameters in terms of "topological and metric relationships"\cite{Kohonen1998} will be retained. 
\\
There are two steps involved in forming a self-organizing map from a raw input data-set\cite{hebbian2007}, respectively to be 1) \textbf{competition} and 2) \textbf{cooperation}. When a set of data is feed into the system sequentially with random shuffle, for each input data point, \textbf{competition} will take place first and, based on a pre-defined cost function, one of the neurons on the output layer with the minimal cost will be selected as a winner; Following the competition, the  \textbf{cooperation} will then take place. Based on a neighborhood function, the winner together with it's neighbor neurons will proceed the learning; while, the neurons outside of the winner's neighbor zone will gain no learning. The purpose of the cooperation step is to increase the like-hood that if a similar input pattern present again, the same group of neurons will become the winner with a higher possibility. Iterate with this strategy on the input data-set over a suitable period, without supervising (providing error to the system), the output layer will simultaneously form a map that contains the similar topological structure as the input data. 



\begin{figure}[h]
  \centering
  \includegraphics[width=8cm]{Picture/AnminPic.JPG}
  \caption{The illustration of the intermediate step to map K inputs (Robots) to M outputs (Targets)\cite{zhu2006neural}. At a time instance, $i$, K neurons out of these $K*M$ possibilities will be selected as the winner. [\textcolor{red}{I do not really understand this!!!, should we have K winner as we only have K inputs or M winner as each neuron should have same possibility to win???}] }
  \Description{More People are using Maven as build tool than Gradle}
\end{figure}


\section{Model and Problem Statement}
Assume there are K homogeneous robots and M targets randomly distributed in a bounded 2D space. Each target requires at least one robot, and all the robots need to be assigned to a target. The cost is defined as the summation of the traveling path length over all the robots to their targets respectively. The goal is to develop a algorithm to find a traveling path for all the robots with the minimal or near minimal cost.
\begin{itemize}
    \item 1D model of K robots and M Targets
    \item 2D model of K robots and M Targets
    \item KD model of K robots and M targets
\end{itemize}



\section{Brute-force Method}
Theoretically, we can solve this problem and find the best solution with the Brute-force method. The time complexity will be bound by $O(M^K)$, Same as the famous traveling sales person (TSP) problem, this is also a NP-complete problem.
\subsection{Entropy of this problem}
\\ According to Shannon's information theory\cite{Shannon1948}, for a random variable $X$ in a finite set $\chi$ with probability distribution $p(x)$ , the  Shannon's entropy can be written as:
\begin{equation}
    H\left( X \right) =  - \sum\limits_{x \in \chi } {p\left( x \right){{\log }_2}\left( {p\left( x \right)} \right)}\label{shannon}
\end{equation}
For each of the robots, it will have at most M choice to choose the target. The total number of possible of arrangement is at most $M^K$. With considering the boundary condition that each of these M target need to have at least one robot, we can reduce the total number of possible of arrangement, $\chi$, as: 

\begin{equation}
\chi  = \underbrace {M\left( {M - 1} \right)\left( {M - 2} \right) \ldots 1}_{M!} \times C_{K - M}^K{M^{K - M}} = M!\frac{{K!}}{{\left( {K - K + M} \right)!\left( {K - M} \right)!}}{M^{K - M}} = \frac{{K!}}{{\left( {K - M} \right)!}}{M^{K - M}}
\end{equation}

%%
%% The acknowledgments section is defined using the "acks" environment
%% (and NOT an unnumbered section). This ensures the proper
%% identification of the section in the article metadata, and the
%% consistent spelling of the heading.
% \begin{acks}
% To Robert, for the bagels and explaining CMYK and color spaces.
% \end{acks}

%%
%% The next two lines define the bibliography style to be used, and
%% the bibliography file.
\bibliographystyle{ACM-Reference-Format}
\bibliography{sample-base}


\end{document}
\endinput
%%
%% End of file `sample-manuscript.tex'.
